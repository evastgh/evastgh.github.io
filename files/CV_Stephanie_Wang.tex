\documentclass[11pt,a4paper,unicode]{moderncv}

\moderncvtheme{fancy}
\definecolor{color1}{rgb}{0.22,0.45,0.45}
\usepackage[utf8]{inputenc}

\usepackage[top=.68in,bottom=.6in,left=.6in, right=.8in]{geometry}          % smaller number -> text takes more space
% \usepackage[scale=0.89]{geometry}
\AtBeginDocument{\recomputelengths}

% The following lines include both bibentry and hyperref packages while resolving the clash
% See https://tex.stackexchange.com/questions/65348/clash-between-bibentry-and-hyperref-with-bibstyle-elsart-harv
\usepackage{bibentry} % using BibTeX for publication list
\makeatletter\let\saved@bibitem\@bibitem\makeatother
\usepackage[unicode]{hyperref} % hyperlinks
\makeatletter\let\@bibitem\saved@bibitem\makeatother

\definecolor{linkcolour}{rgb}{0,0.2,0.6}            % hyperlinks setup
\hypersetup{colorlinks,breaklinks,urlcolor=linkcolour, linkcolor=linkcolour}

\usepackage{enumitem} %customization for enumerate

\firstname{Stephanie}
\familyname{Wang \vspace*{-5mm}}
\email{evast@g.ucla.edu}
\homepage{stephaniewang.page}

\renewcommand{\emailsymbol}{}
\renewcommand{\homepagesymbol}{}
\setlength{\hintscolumnwidth}{25mm}
% \setlength{\separatorcolumnwidth}{3mm}
\setlength{\maincolumnwidth}{20mm}

\nopagenumbers{}                    

\begin{document}
\maketitle

\vspace*{-2mm}
\section{Brief intro}
\cvline{}{
    I have worked in and published papers in the fields of \textbf{geometric deep learning}, \textbf{geometry optimization}, and \textbf{physics-based simulation} of all kinds of materials (solids, fluids, waves, cloth, hair, friction / contact, deformable bodies, plasticity, visco-elasto-plasticity, and fracture / damage). My research aims to provide mathematical and engineering solutions to bring the virtual world to reality by building robust and high-performance computer graphics algorithms. 
    % \vadjust{\vspace{2pt}}\newline
    % \textbf{Geometric deep learning: }
    % Generative AI has demonstrated huge success in the space of 2D content generation, specifically with text-to-image models. To achieve similar success in 3D content generation, researchers need to build machine learning models that are geometric in nature. Existing neural representations of 3D geometry still face significant challenges, \textit{e.g.} the discontinuity introduced by topological changes. Geometric measure theory is a branch of mathematics that treats 3D geometries as vector fields, generalizing the levelset- and SDF-based models, and has great potential in solving the 3D representation problem. I am interested in finding the bridge between the (sometimes esoteric) math literature and the increasing demand for a modern geometric deep learning model. 
    % \vadjust{\vspace{2pt}}\newline
    % \textbf{Physics modeling: }
    % Physical phenomena like solids, fluids, friction, fracture, and more are crucial in building a appealing virtual world. The demand for high-performance, visually-appealing, and physically-accurate simulation methods are growing across industries. Many face a specific set of constraints and require domain-knowledge to build an optimized simulation pipeline. Film industry demands physically-plausible solutions while preserving art-directability. Game industry requires real-time results on varying hardwares. Manufacturing industry uses simulation to assist design and development as well as to cut costs in manufacturing samples and testing. I have deep understanding in the physics of all of systems that might play a role in any simulated scene, as well as research experience (and publications) in simulating the interaction and coupling of several systems. My experience in developing and maintaining large-scale C++ libraries can also come in handy when building pipelines for high-resolution (large data) or real-time (fast computation) simulations. 
    % \vadjust{\vspace{2pt}}\newline
    % \textbf{Geometry optimization: }
    % Many industrial processes involve geometry. Finding the optimal shape for a new product that balances both manufacturing costs and performance is often the top challenge. Building a differentiable forward simulation pipeline alone is not enough. One also needs to choose the right optimization method in order to obtain the optimal geometry. Mathematical insights for the geometry of the problem can often help boost convergence rate of the optimization problem by \textit{magnitudes}, and sometimes even convert a nonconvex problem into a convex one. 
    % \vadjust{\vspace{2pt}}\newline
    % In addition to my 5-year experience in developing and maintaining C++ code, I also have 4 years of experience using Houdini to implement prototypes and research codes. Developing in Houdini gives short turn-around time (3-5x faster than using C++) and quick visualization during early stages of a research project. 
    %
    %
    % SOME RANDOM DRAFT
    % A physically-plausible simulation engine can enhance the artistic expression and story-telling in animation and games, as well as provide a fast and cheap 
    % I hope to build plausible physical dynamics in animation and games to enable artistic expression and compelling story-telling.
    % The impact that I'm seeking is to enhance users' artistic expression by improving physical dynamics in animation and games to create a more compelling story-telling environment, as well as optimizing the industrial manufacturing processes by involving novel graphics technology in the development phase. 
    % I am also seeking to bridge my decade's worth understanding in the mathematical models in graphics pipeline to machine learning models to enable the next generation of generative AI and content generation. 
    % I have published many scientific papers as well as giving research talks
    % My track record of publications and collaborations with researchers in both academia and industry serves to proof my skills in technical communication and team-building.
    % Points: team-building, Houdini for fast prototyping, fluent technical communication skills.
    % END RANDOM DRAFT
}
\cvline{}{I am currently looking for Research Scientist / Research Engineer / Software Engineer positions that start in \textbf{January 2024}. }


\vspace*{-2mm}
\section{Education}
%\cventry{years}{degree/job title}{institution/employer}{localization}{grade}{description}
\cventry{2014-2020}{Ph.D. and M.S. in Mathematics}{UCLA}{}{Eugene V. Cota-Robles Fellow}{Committee: \href{http://seas.ucla.edu/sofia/jeff/index.php}{Jeffrey D. Eldredge}, \href{https://wotaoyin.mathopt.com/}{Wotao Yin}, \href{https://www.math.ucla.edu/~lvese/}{Luminita Aura Vese}, and \href{https://www.math.ucla.edu/~jteran/}{Joseph M. Teran} (advisor)}
\cventry{2009-2013}{B.S. in Mathematics}{National Taiwan University}{}{\textit{magna cum laude}}{}

\vspace*{-2mm}
\section{Experience}
\subsection{Research}
\cventry{2020-present}{Postdoc -- with Prof. \href{https://cseweb.ucsd.edu/~alchern/}{Albert Chern}}{UCSD}{}{San Diego, CA}{Geometry processing and physical simulation using mathematical insights from geometric measure theory, exterior calculus, partial differential equations, and optimization theory. 
    Developing in Houdini and Python. 
    Mentored students: 
    \href{https://sinabiz.github.io/}{Mohammad Sina Nabizadeh},
    \href{https://shiyang-jia.com/}{Shiyang Jia},
    \href{https://chadmckell.com/}{Chad McKell},
    \href{https://yhesper.github.io/}{Hang Yin},
    \href{https://www.linkedin.com/in/baichuan-wu-584871142}{Baichuan Wu}.
}
\cvline{}{(Note: I took a full-time medical leave between Feb-Aug 2022 to recover from an acute illness.)}
\cventry{2019-2020}{Ph.D. Study -- with Prof. \href{https://www.math.ucla.edu/~wgangbo/}{Wilfrid Gangbo}}{UCLA}{}{Los Angeles, CA}{Regularity theory for minimizers of polyconvex functionals related to Navier-Stokes equation. }
\cventry{2019 summer}{Exchange Study -- with Prof. \href{https://people.epfl.ch/johan.gaume}{Johan Gaume}}{EPFL}{}{Lausanne, Switzerland}
{Simulations and data analysis of snow and tire interaction, avalanche release, and snow micro-structure.}
\cventry{2016-2019}{Ph.D Study -- with Prof. \href{https://www.math.ucla.edu/~jteran/}{Joseph Teran}}{UCLA}{}{Los Angeles, CA}
{Physics-based simulations of various materials with Material Point Method and Finite Element Method, using continuum mechanics, convex and nonconvex optimization technique, numerical analysis, parallel computing, developing in C++ and Houdini. }
\cventry{2013-2014}{Research Assistant -- with Prof. \href{https://scholar.nycu.edu.tw/en/persons/wen-wei-lin}{Wen-Wei Lin}}{NCTU}{}{Hsinchu, Taiwan}{Generalized eigenvalue problems using MATLAB programming.}

\subsection{Industry}
\cventry{2018 summer}{Tech Intern}{Walt Disney Animation Studio}{}{Burbank, CA}
{R\&D for pioneering simulation technology in animated feature films, teaming with VFX artists and developing in C++ and HDK. }

\subsection{Teaching}
\cventry{2020}{Assistant Adjunct Professor}{UCLA Math Dept}{}{Los Angeles, CA (virtual)}
{Taught remote classes for upper and lower division undergratuate courses: Machine Learning (Math156) and Calculus of Several Variables (Math32A).}
\cventry{2019 spring}{Graduate Student Instructor}{UCLA Math Dept}{}{Los Angeles, CA}
{Taught course: Linear Algebra and Applications (Math33A).}
\cventry{2015-2020}{Teaching Assistant}{UCLA Math Dept}{}{Los Angeles, CA}
{Led discussion sessions and graded homework/exams for 11 undergraduate and graduate level courses: linear algebra and introduction to mathematical proofs (Math 115A), undergrad- and grad-level numerical methods (Math 151B, 269A), introductory, intermediate, and advanced C++ programming (PIC 10A, 10B, 10C).}
% \cventry{2014 summer}{Course Organizer}{Formosan Summer School on Logic, Language, and Computation}{}{}{} % {Organized 2014 Formosan Summer School on Logic, Language, and Computation.}

\vspace*{-2mm}
\section{Awards}
\cventry{May 2022}{Rising Stars in Computer Graphics Research}{WiGRAPH}{}{}{}
\cventry{Jul 2019}{Best Paper Award}{ACM SIGGRAPH/Eurographics Symposium on Computer Animation}{}{}{}
\cventry{Sep 2014}{Eugene V. Cota-Robles Fellowship}{UCLA}{}{}{}
\cventry{Jun 2013}{Dean's Award, College of Science}{National Taiwan University}{}{}{}
\cventry{Aug 2012}{Bronze Medal in Applied and Computational Mathematics}{S.T. Yau College Student Mathematics Contest}{}{}{}

\vspace*{-2mm}
\section{Preprint}
\cvline{}{
    \href{https://arxiv.org/abs/2305.08033}{\textbf{Wave Simulations in Infinite Spacetime}}
    \small\newline Chad McKell, Mohammad Sina Nabizadeh, \underline{Stephanie Wang}, Albert Chern
}
\vspace*{-2mm}
\section{Publications}
% \nobibliography{PublicationsAndCitations}
% \bibliographystyle{ACM-Reference-Format}
% \begin{enumerate}
%     \item \bibentry{Nabizadeh:2022}
%     \item \bibentry{Palmer:2022}
%     \item \bibentry{Carlen:2022}
%     \item \bibentry{Blatny:2021}
%     \item \bibentry{Wang:2021}
%     \item \bibentry{Wang:2020}
%     \item \bibentry{Ding:2019}
%     \item \bibentry{Han:2019}
%     \item \bibentry{Wang:2019}
% \end{enumerate}
\cvline{}{
    \href{https://dl.acm.org/doi/abs/10.1145/3606938?casa_token=UwuWRTVEYPwAAAAA:UwXefvIfIPZOpOFHiNRPazPUFL9Lw66liJzLsIopkxayiNgWgNWluiJWHad9NPjXKjuW0Re43rU}{\textbf{Physical Cyclic Animations}}
    \small\newline Shiyang Jia, \underline{Stephanie Wang}, Tzu-Mao Li, Albert Chern
    \newline Proceedings of the ACM on Computer Graphics and Interactive Techniques (SCA 2023) 
}
\cvline{}{
    \href{https://dl.acm.org/doi/10.1145/3587423.3595525}{\textbf{Exterior Calculus in Graphics: Course Notes for a SIGGRAPH 2023 Course}}
    \small\newline\underline{Stephanie Wang}, Mohammad Sina Nabizadeh, Albert Chern
    \newline {SIGGRAPH '23: ACM SIGGRAPH 2023 Courses} 
}
\cvline{}{
    \href{https://dl.acm.org/doi/10.1145/3592402}{\textbf{Fluid Cohomology}}
    \small\newline Hang Yin, Mohammad Sina Nabizadeh, Baichuan Wu, \underline{Stephanie Wang}, Albert Chern
    \newline {ACM Transactions on Graphics (SIGGRAPH 2023)} 
}
\cvline{}{
    \href{https://dl.acm.org/doi/10.1145/3528223.3530120}{\textbf{Covector Fluids}}
    \small\newline Mohammad Sina Nabizadeh, \underline{Stephanie Wang}, Ravi Ramamoorthi, Albert Chern
    \newline {ACM Transactions on Graphics (SIGGRAPH 2022)} 
}
\cvline{}{
    \href{https://openaccess.thecvf.com/content/CVPR2022/papers/Palmer_DeepCurrents_Learning_Implicit_Representations_of_Shapes_With_Boundaries_CVPR_2022_paper.pdf}{\textbf{DeepCurrents: Learning Implicit Representations of Shapes with Boundaries}}
    \small\newline David Palmer, Dmitriy Smirnov, \underline{Stephanie Wang}, Albert Chern, Justin Solomon
    \newline {Proceedings of the IEEE/CVF Conference on Computer Vision and Pattern Recognition (CVPR 2022)}
}
\cvline{}{
    \href{https://www.cambridge.org/core/journals/network-science/article/role-detection-in-bicyclesharing-networks-using-multilayer-stochastic-block-models/5D73728650C5C3E2DB9455FCDF46F0E2}{\textbf{Role Detection in Bicycle-Sharing Networks Using Multilayer Stochastic Block Models}}
    \small\newline Jane Carlen, Jaume de Dios Pont, Cassidy Mentus, Shyr-Shea Chang, \underline{Stephanie Wang}, Mason A. Porter
    \newline {Network Science, 2022}
}
\cvline{}{
    \href{https://dl.acm.org/doi/10.1145/3450626.3459781}{\textbf{Computing minimal surfaces with differential forms}}
    \small\newline \underline{Stephanie Wang} and Albert Chern
    \newline {ACM Transactions on Graphics (SIGGRAPH 2021)} 
}
\cvline{}{
    \href{https://www.sciencedirect.com/science/article/pii/S0266352X21002822}{\textbf{Computational micromechanics of porous brittle solids}}
    \small\newline Lars Blatny, Henning Löwe, \underline{Stephanie Wang}, Johan Gaume
    \newline {Computers and Geotechnics, 2021} 
}
\cvline{}{
    \href{https://www.proquest.com/docview/2389768700?pq-origsite=gscholar&fromopenview=true}{\textbf{A Material Point Method for Elastoplasticity with Ductile Fracture and Frictional Contact}}
    \small\newline \underline{Stephanie Wang}
    \newline UCLA Doctoral Dissertation, 2020
}
\cvline{}{
    \href{https://dl.acm.org/doi/10.1145/3355089.3356537}{\textbf{A thermomechanical material point method for baking and cooking}}
    \small\newline Mengyuan Ding, Xuchen Han, \underline{Stephanie Wang}, Theodore F. Gast, Joseph M. Teran
    \newline ACM Transactions on Graphics (SIGGRAPH Asia 2019) 
}
\cvline{}{
    \href{https://dl.acm.org/doi/10.1145/3340258}{\textbf{A Hybrid Material Point Method for Frictional Contact with Diverse Materials}} 
    \small\newline Xuchen Han, Theodore F. Gast, Qi Guo, \underline{Stephanie Wang}, Chenfanfi Jiang, Joseph M. Teran
    \newline Proceedings of the ACM on Computer Graphics and Interactive Techniques (SCA 2019) 
}
\cvline{}{
    \href{https://dl.acm.org/doi/10.1145/3340259}{\textbf{Simulation and Visualization of Ductile Fracture with the Material Point Method}}
    \small\newline \underline{Stephanie Wang}, Mengyuan Ding, Theodore F. Gast, Leyi Zhu, Steven Gagniere, Chenfanfu Jiang, Joseph M. Teran
    \newline Proceedings of the ACM on Computer Graphics and Interactive Techniques (\textbf{SCA 2019 Best Paper}) 
}

\vspace*{-2mm}
\section{Invited talks}
% \cventry[spacing]{years}{degree/job title}{institution/employer}{localization}{optionnal: grade/...}{optional: comment/job description}
% \cvline{}{\textcolor{color1}{Conferences and workshops}}
% \subsection{ }
\cvline{}{\textcolor{color1}{{\subsectionfont Conferences / Workshops}}}
\cventry{Aug 2023}{ICIAM 2023}{}{Tokyo, Japan}{}{}
\cventry{Aug 2023}{SIGGRAPH 2023}{}{Los Angeles, CA}{}{}
\cventry{Sep 2021}{Geometry Workshop in Obergurgl 2021}{}{Obergurgl, Austria}{}{}
\cventry{Aug 2021}{SIGGRAPH 2021}{}{(virtual)}{}{}
\cventry{Aug 2019}{SCA 2019}{}{Los Angeles, CA}{}{}
%
%
% \subsection{ }
\cvline{}{\textcolor{color1}{\emph{Colloquia / Seminars}}}
\cventry{Feb 2022}{NCSU}{}{Raleigh, NC (virtual)}{}{}{} % Geometry and Topology Seminar
\cventry{Nov 2021}{MIT}{}{Cambridge, MA}{}{}{} % CSAIL talk
\cventry{Nov 2021}{Autodesk}{}{(virtual)}{}{}
\cventry{Nov 2021}{Online Seminar Geometric Analysis}{}{(virtual)}{}{}
\cventry{Oct 2021}{Toronto Geometry Colloquium}{}{Toronto, ON (virtual)}{}{}
% \cventry{Sep 2021}{Geometry Workshop in Obergurgl}{}{Obergurgl, Austria}{}{}
% \cventry{Aug 2021}{SIGGRAPH}{}{(virtual)}{}{}
\cventry{Apr 2021}{UCSD (CSE Vis-Comp)}{}{San Diego, CA (virtual)}{}{}
\cventry{Jan 2021}{UCSD (CCoM)}{}{San Diego, CA (virtual)}{}{}
\cventry{Dec 2020}{CMU}{}{Pittsburgh, PA (virtual)}{}{}
\cventry{May 2020}{GAMES Webinar}{}{(virtual)}{}{}
\cventry{Nov 2019}{College of the Holy Cross}{}{Worcester, MA (virtual)}{}{}
\cventry{Sep 2019}{Inria Grenoble-Rh\^one-Alpes}{}{Grenoble, France}{}{}
\cventry{Aug 2019}{ETH Z\"urich}{}{Z\"urich, Switzerland}{}{}

\subsection{ }
\cvline{}{\textcolor{color1}{\emph{Graduate Student Seminars}}}
\cventry{Aug 2019}{EPFL}{}{Lausanne, Switzerland}{}{}
% \cventry{Aug 2019}{SCA}{}{Los Angeles, CA}{}{}
\cventry{Nov 2018}{UCLA}{}{Los Angeles, CA}{}{}

\vspace*{-2mm}
\section{Services}
\cventry{2022-present}{Program committee}{ACM SIGGRAPH, Eurographics}{}{}{
ACM SIGGRAPH 2023 poster jury 
and Eurographics 2023 short paper program committee 
}
\cventry{2021-present}{Reviewer}{ACM SIGGRAPH North America, ACM SIGGRAPH Asia, Eurographics, ICIAM, IEEE TVCG}{}{}{Reviewed technical papers in areas including geometry processing, physical simulation, and scientific computing.}
\cventry{2021}{Research project mentor}{MIT Summer Geometry Initiative}{}{}{Designed a research project and advised undergraduate fellows on minimal surfaces using both Lagrangian and Eulerian representations.}
\cventry{2017-2020}{Math Dept Representative}{Graduate Student Association, UCLA}{}{}{Advocated for student rights in campus-level organizations and organized cross-department social events. }
\cventry{2015-2020}{Volunteer}{AWiSE STEM Day, Explore Your Universe}{}{}{Presented interactive math booth in annual science fair designated for middle school girls and general public. }
\cventry{2016-2018}{Chief Organizer}{Women in Math, UCLA}{}{}{Organized social and volunteering events, represented and advocated for women in math dept. }
\cventry{2017}{Creator}{Women in Math Mentorship Program, UCLA}{}{}{Negotiated for fundings and created the program that hosts regular mixers for undergraduate and graduate fellows to increase connection, awareness, and mentorship.}
\cventry{2016-2018}{Fellow Mentor}{California Teach, UCLA}{}{}{Mentored and advised Math and Statistics undergraduate students from underrepresented demographics. }
\cventry{2012-2013}{Vice President}{Lambda Club, National Taiwan University}{}{}{Organized academic and social events and grew the community from 3 people to 30+ during my service. }


\vspace*{-2mm}
\section{Skills}
\cvline{Programming}{C++ (Eigen, tbb), Python (PyTorch, SciPy), lua, MATLAB (CVX), \LaTeX, zsh}
\cvline{Tools}{Houdini, Vim, git, gdb, valgrind}
\cvline{Math}{Optimization, differential geometry, numerical and theoretical PDEs, scientific computing}
\cvline{Languages}{English and Mandarin Chinese (bilingual)}{}
\cvline{Hobbies}{Rock climbing, hiking, cooking}

% Note using moderncv + BibTeX because want to add hyperlinks manually
% \nocite{*}
% \bibliographystyle{plain}
% \nobibliography{publications}
% \renewcommand{\refname}{Selected Publications}

% \vspace*{-2mm}
\begin{center}
    Last updated: \today.
\end{center}

\end{document}
