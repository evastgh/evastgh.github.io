\documentclass[12pt,a4paper]{article}

% \usepackage[utf8]{inputenc}
% \DeclareUnicodeCharacter{1D12A}{\doublesharp}
% \DeclareUnicodeCharacter{2693}{\anchor}
% \usepackage{dingbat}
% \DeclareRobustCommand\dash\unskip\nobreak\thinspace{\textemdash\allowbreak\thinspace\ignorespaces}
\usepackage[top=1in, bottom=1in, left=1in, right=1in, headheight=15pt]{geometry}
%\usepackage{fullpage}

\usepackage{tabularx}
\usepackage{multirow}
\usepackage{makecell}
\usepackage{fancyhdr}\pagestyle{fancy}\rhead{Stephanie Wang}\lhead{Research Statement}

\usepackage{amsmath,amssymb,amsthm,amsfonts,microtype,stmaryrd,mathtools}
	%{wasysym,yhmath}

\usepackage[usenames,dvipsnames]{xcolor}
\newcommand{\blue}[1]{\textcolor{blue}{#1}}
\newcommand{\red}[1]{\textcolor{red}{#1}}
\newcommand{\gray}[1]{\textcolor{gray}{#1}}
\newcommand{\fgreen}[1]{\textcolor{ForestGreen}{#1}}

\usepackage{mdframed}
	%\newtheorem{mdexample}{Example}
	\definecolor{warmgreen}{rgb}{0.8,0.9,0.85}
	% --Example:
	% \begin{center}
	% \begin{minipage}{0.7\textwidth}
	% \begin{mdframed}[backgroundcolor=warmgreen, 
	% skipabove=4pt,skipbelow=4pt,hidealllines=true, 
	% topline=false,leftline=false,middlelinewidth=10pt, 
	% roundcorner=10pt] 
	%%%% --CONTENTS-- %%%%
	% \end{mdframed}\end{minipage}\end{center}	

\usepackage{graphicx} \graphicspath{{}}
	% --Example:
	% \includegraphics[scale=0.5]{picture name}
%\usepackage{caption} %%% --some awful package to make caption...

\usepackage[unicode]{hyperref}\hypersetup{linktocpage,colorlinks}\hypersetup{citecolor=black,filecolor=black,linkcolor=black,urlcolor=blue,breaklinks=true}

%%% --Text Fonts
%\usepackage{times} %%% --Times New Roman for LaTeX
%\usepackage{fontspec}\setmainfont{Times New Roman} %%% --Times New Roman; XeLaTeX only

%%% --Math Fonts
\renewcommand{\v}[1]{\ifmmode\mathbf{#1}\fi}
\renewcommand{\b}[1]{\ifmmode\boldsymbol{#1}\fi}
%\renewcommand{\mbf}[1]{\mathbf{#1}} %%% --vector
%\newcommand{\ca}[1]{\mathcal{#1}} %%% --"bigO"
%\newcommand{\bb}[1]{\mathbb{#1}} %%% --"Natural, Real numbers"
%\newcommand{\rom}[1]{\romannumeral{#1}} %%% --Roman numbers

%%% --Quick Arrows
\newcommand{\ra}[1]{\ifnum #1=1\rightarrow\fi\ifnum #1=2\Rightarrow\fi\ifnum #1=3\Rrightarrow\fi\ifnum #1=4\rightrightarrows\fi\ifnum #1=5\rightleftarrows\fi\ifnum #1=6\mapsto\fi\ifnum #1=7\iffalse\fi\fi\ifnum #1=8\twoheadrightarrow\fi\ifnum #1=9\rightharpoonup\fi\ifnum #1=0\rightharpoondown\fi}

%\newcommand{\la}[1]{\ifnum #1=1\leftarrow\fi\ifnum #1=2\Leftarrow\fi\ifnum #1=3\Lleftarrow\fi\ifnum #1=4\leftleftarrows\fi\ifnum #1=5\rightleftarrows\fi\ifnum #1=6\mapsfrom\ifnum #1=7\iffalse\fi\fi\ifnum #1=8\twoheadleftarrow\fi\ifnum #1=9\leftharpoonup\fi\ifnum #1=0\leftharpoondown\fi}

%\newcommand{\ua}[1]{\ifnum #1=1\uparrow\fi\ifnum #1=2\Uparrow\fi}
%\newcommand{\da}[1]{\ifnum #1=1\downarrow\fi\ifnum #1=2\Downarrow\fi}

%%% --Special Editor Config
\renewcommand{\ni}{\noindent}
\newcommand{\onum}[1]{\raisebox{.5pt}{\textcircled{\raisebox{-1pt} {#1}}}}

\newcommand{\claim}[1]{\underline{``{#1}":}}

\renewcommand{\l}{\left}\renewcommand{\r}{\right}\newcommand{\m}{\middle}

\newcommand{\casebrak}[4]{\left \{ \begin{array}{ll} {#1},&{#2}\\{#3},&{#4} \end{array} \right.}
%\newcommand{\ttm}[4]{\l[\begin{array}{cc}{#1}&{#2}\\{#3}&{#4}\end{array}\r]} %two-by-two-matrix
%\newcommand{\tv}[2]{\l[\begin{array}{c}{#1}\\{#2}\end{array}\r]}

\def\dps{\displaystyle}

\let\italiccorrection=\/
\def\/{\ifmmode\expandafter\frac\else\italiccorrection\fi}


%%% --General Math Symbols
\def\bc{\because}
\def\tf{\therefore}

%%% --Frequently used OPERATORS shorthand
\newcommand{\INT}[2]{\int_{#1}^{#2}}
\newcommand{\Int}[1]{\int_{#1}}
% \newcommand{\UPINT}{\bar\int}
% \newcommand{\UPINTRd}{\overline{\int_{\bb R ^d}}}
\newcommand{\SUM}[2]{\sum\limits_{#1}^{#2}}
\newcommand{\PROD}[2]{\prod\limits_{#1}^{#2}}
\newcommand{\CUP}[2]{\bigcup\limits_{#1}^{#2}}
\newcommand{\CAP}[2]{\bigcap\limits_{#1}^{#2}}
% \newcommand{\SUP}[1]{\sup\limits_{#1}}
% \newcommand{\INF}[1]{\inf\limits_{#1}}
\newcommand{\pd}[2]{\frac{\partial{#1}}{\partial{#2}}}
\def\tr{\text{tr}}

\renewcommand{\o}{\circ}
\newcommand{\x}{\times}
\newcommand{\ox}{\otimes}

\newcommand\ie{{\it i.e. }}
\newcommand\wrt{{w.r.t. }}
\newcommand\dom{\mathbf{dom\:}}

%%% --Frequently used VARIABLES shorthand
\newcommand{\R}{\ifmmode\mathbb R\fi}
\newcommand{\N}{\ifmmode\mathbb N\fi}
\newcommand{\T}{\ifmmode\mathbb T\fi}
\renewcommand{\O}{\mathcal{O}}
\newcommand{\w}{\wedge}
\newcommand{\ome}{\omega}
\newcommand{\lam}{\lambda}
\newcommand{\im}{\mbox{im}}

\newcommand{\B}{\mathcal B}

%============================
% Conjugate, Double bracket
%============================

\makeatletter
\DeclareFontFamily{OMX}{MnSymbolE}{}
\DeclareSymbolFont{MnLargeSymbols}{OMX}{MnSymbolE}{m}{n}
\SetSymbolFont{MnLargeSymbols}{bold}{OMX}{MnSymbolE}{b}{n}
\DeclareFontShape{OMX}{MnSymbolE}{m}{n}{
    <-6>  MnSymbolE5
   <6-7>  MnSymbolE6
   <7-8>  MnSymbolE7
   <8-9>  MnSymbolE8
   <9-10> MnSymbolE9
  <10-12> MnSymbolE10
  <12->   MnSymbolE12
}{}
\DeclareFontShape{OMX}{MnSymbolE}{b}{n}{
    <-6>  MnSymbolE-Bold5
   <6-7>  MnSymbolE-Bold6
   <7-8>  MnSymbolE-Bold7
   <8-9>  MnSymbolE-Bold8
   <9-10> MnSymbolE-Bold9
  <10-12> MnSymbolE-Bold10
  <12->   MnSymbolE-Bold12
}{}
\let\llangle\@undefined
\let\rrangle\@undefined
\DeclareMathDelimiter{\llangle}{\mathopen}%
                     {MnLargeSymbols}{'164}{MnLargeSymbols}{'164}
\DeclareMathDelimiter{\rrangle}{\mathclose}%
                     {MnLargeSymbols}{'171}{MnLargeSymbols}{'171}
\makeatother

\title{Research Statement}
\author{Stephanie Wang}
\date{\today}

% The following lines include both bibentry and hyperref packages while resolving the clash
% See https://tex.stackexchange.com/questions/65348/clash-between-bibentry-and-hyperref-with-bibstyle-elsart-harv
\usepackage{bibentry} % using BibTeX for publication list
\makeatletter\let\saved@bibitem\@bibitem\makeatother
\makeatletter\let\@bibitem\saved@bibitem\makeatother

%%%%%%%%%%%%%%%%%%%%%%%%%%%%%%%%%%%%%%%%%%%%%%%%%%%%%%%%%%%%%%%%%%%%%%%%%%%%%%%%%%%%%%%%%%%%%%%%%%%%%%%%%%%%%%%%%%%%%%%%%%%%%%%%%%%%%%%%%%%%%%%%%%%%%%%%%%%%%%%%%%%%%%%%%%%%%%%%%%%%%%%%%%%%%
\begin{document}
\maketitle

My research interests lie in partial differential equations and optimization with applications in computer graphics, specifically physics simulation and geometry processing. 
I find myself gravitating towards research topics that have a real-world impact that also allow me to apply my pure math training. 

There is an abundance of physics models involved in building a virtual world. 
High-fidelity simulations of physical phenomena are becoming more important than ever due to the popularity of 4K cinematography. 
Researchers have proposed mathematical models for various materials including snow \cite{stomakhin2013material}, sand \cite{klar2016drucker}, smoke \cite{zehnder2018advection}. 
Due to the highly nonlinear equations involved in these models, the research focuses the discretization models and preserving important physical quantities including mass and momentum. 
For graphics applications, a unique challenge arises with the modeling of contact forces. 
The current state of the art either uses interpolation between a grid and particles to remove self-penetrating motions \cite{jiang2017anisotropic}, or introduces barriers between material positions and solves a large optimization problem \cite{li2020incremental}. 
There are still lots and lots of mathematical properties that are awaiting exploration in this field. 
In my recent accepted work \cite{Nabizadeh:2022}, we demonstrated that a discretization of Euler’s equation can generate vivid vortex dynamics provided that it respects Kelvin’s circulation theorem. 
I would like to further study the profound theory of each physical phenomenon and develop robust algorithms based on the mathematical and physical properties. 

I became interested in geometry processing while researching my 2019 work on the simulation and visualization of fracturing materials \cite{Wang:2019,Ding:2019}. 
In order to create crispy edges around a fracture interface, I had to perform topology changes on meshes that were highly tedious. 
In my more recent work \cite{Wang:2021}, we explored a different geometry representation called “current” that abandons the traditional vertex connectivities altogether. 
The classical geometric measure theory provides valuable insights that convert the traditionally nonconvex problem of minimal surfaces into a convex one using the current representation. 
The current representation enjoys several properties including linear boundary operations, convex combination in the geometry configuration space, and differentiability. 
The differentiability also allows us to explore the neural network discretization with promising results in \cite{Palmer:2022}. 
This is another example of mathematical analysis serving to create better discretization and algorithms for computations in graphics.

\bibliography{PublicationsAndCitations}
\bibliographystyle{ACM-Reference-Format}
\end{document}
