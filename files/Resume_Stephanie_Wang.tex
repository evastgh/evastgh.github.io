\documentclass[11pt,a4paper,unicode]{moderncv}

\moderncvtheme{fancy}
\definecolor{color1}{rgb}{0.22,0.45,0.45}
\usepackage[utf8]{inputenc}

\usepackage[top=.7in,bottom=.6in,left=.6in, right=.8in]{geometry}          % smaller number -> text takes more space
% \usepackage[scale=0.89]{geometry}
\AtBeginDocument{\recomputelengths}

% The following lines include both bibentry and hyperref packages while resolving the clash
% See https://tex.stackexchange.com/questions/65348/clash-between-bibentry-and-hyperref-with-bibstyle-elsart-harv
\usepackage{bibentry} % using BibTeX for publication list
\makeatletter\let\saved@bibitem\@bibitem\makeatother
\usepackage[unicode]{hyperref} % hyperlinks
\makeatletter\let\@bibitem\saved@bibitem\makeatother

\definecolor{linkcolour}{rgb}{0,0.2,0.6}            % hyperlinks setup
\hypersetup{colorlinks,breaklinks,urlcolor=linkcolour, linkcolor=linkcolour}

\usepackage{enumitem} %customization for enumerate

\firstname{Stephanie}
\familyname{Wang \vspace*{-4mm}}
\email{evast@g.ucla.edu}
\homepage{stephaniewang.page}

\renewcommand{\emailsymbol}{}
\renewcommand{\homepagesymbol}{}
\setlength{\hintscolumnwidth}{25mm}
% \setlength{\separatorcolumnwidth}{3mm}
\setlength{\maincolumnwidth}{20mm}

\nopagenumbers{}                    

\begin{document}
\maketitle

\vspace*{-2mm}
\section{Education}
%\cventry{years}{degree/job title}{institution/employer}{localization}{grade}{description}
\cventry{2014-2020}{Ph.D. and M.S. in Mathematics}{UCLA}{}{Eugene V. Cota-Robles Fellow}{}
\cventry{2009-2013}{B.S. in Mathematics}{National Taiwan University}{}{\textit{magna cum laude}}{}

\vspace*{-2mm}
\section{Positions}
\subsection{Research}
\cventry{2020-present}{Postdoc -- with Prof. \href{https://cseweb.ucsd.edu/~alchern/}{Albert Chern}}{UCSD}{}{San Diego, CA}{Geometry processing and physical simulation using mathematical insights from geometric measure theory, exterior calculus, partial differential equations, and optimization theory. Mentored PhD students: 
    \href{https://sinabiz.github.io/}{Mohammad Sina Nabizadeh},
    \href{https://www.linkedin.com/in/shiyang-jia-a4aa6b174}{Shiyang Jia},
    \href{https://chadmckell.com/}{Chad McKell},
    \href{https://yhesper.github.io/}{Hang Yin}.
}
\cventry{2019-2020}{Ph.D. Study -- with Prof. \href{https://www.math.ucla.edu/~wgangbo/}{Wilfrid Gangbo}}{UCLA}{}{Los Angeles, CA}{Regularity theory for minimizers of polyconvex functionals related to Navier-Stokes equation. }
\cventry{2019 summer}{Exchange Study -- with Prof. \href{https://people.epfl.ch/johan.gaume}{Johan Gaume}}{EPFL}{}{Lausanne, Switzerland}
{Simulations, post-processing, and data analysis of snow and tire interaction; general consultation at the Snow and Avalanche Simulation Laboratory. }
\cventry{2016-2019}{Ph.D Study -- with Prof. \href{https://www.math.ucla.edu/~jteran/}{Joseph Teran}}{UCLA}{}{Los Angeles, CA}
{Physics-based simulations of various materials with Material Point Method and Finite Element Method, using continuum mechanics, convex and nonconvex optimization technique, numerical analysis, parallel computing, developing in C++ and Houdini. }
% \cventry{2013-2014}{Research Assistant -- with Prof. \href{https://scholar.nycu.edu.tw/en/persons/wen-wei-lin}{Wen-Wei Lin}}{NCTU}{}{Hsinchu, Taiwan}{Generalized eigenvalue problems using MATLAB programming.}

\subsection{Industry}
\cventry{2018 summer}{Tech Intern}{Walt Disney Animation Studio}{}{Burbank, CA}
{R\&D for pioneering simulation technology in animated feature films, teaming with VFX artists and developing in C++ and HDK. }

\subsection{Teaching}
\cventry{2019-2020}{Assistant Adjunct Professor / Instructor}{UCLA Math Dept}{}{Los Angeles, CA}
{Taught three courses: linear algebra, machine learning (remote) and multivariable calculus (remote). }
\cventry{2015-2020}{Teaching Assistant}{UCLA Math Dept}{}{Los Angeles, CA}
{Taught 11 courses: linear algebra and intro to mathematical proofs, undergrad and grad level numerical methods, intro, intermediate, and advanced C++ programming.}
% \cventry{2014 summer}{Course Organizer}{Formosan Summer School on Logic, Language, and Computation}{}{}{} % {Organized 2014 Formosan Summer School on Logic, Language, and Computation.}


\vspace*{-2mm}
\section{Skills}
\cvline{Programming}{C++ (Eigen, tbb, gdb, valgrind), lua, MATLAB (CVX), \LaTeX, Vim, git, Houdini}
\cvline{Math}{Optimization, differential geometry, numerical and theoretical PDEs, scientific computing}
\cvline{Languages}{English and Mandarin Chinese - bilingual proficiency}{}

% Note using moderncv + BibTeX because want to add hyperlinks manually
% \nocite{*}
% \bibliographystyle{plain}
% \nobibliography{publications}
% \renewcommand{\refname}{Selected Publications}

\vspace*{-2mm}
\section{Selected Publications}
\cvline{}{
    \href{https://dl.acm.org/doi/10.1145/3528223.3530120}{Covector fluids}. 
    Mohammad Sina Nabizadeh, \underline{Stephanie Wang}, Ravi Ramamoorthi, and Albert Chern. SIGGRAPH 2022.
}
\cvline{}{
    \href{https://openaccess.thecvf.com/content/CVPR2022/papers/Palmer_DeepCurrents_Learning_Implicit_Representations_of_Shapes_With_Boundaries_CVPR_2022_paper.pdf}{DeepCurrents: Learning implicit representations of shapes with boundaries}. 
    David Palmer, Dmitriy Smirnov, \underline{Stephanie Wang}, Albert Chern, and Justin Solomon. CVPR 2022.
}
\cvline{}{
    \href{https://dl.acm.org/doi/10.1145/3450626.3459781}{Computing minimal surfaces with differential forms}. 
    \underline{Stephanie Wang} and Albert Chern. SIGGRAPH 2021.
}
\cvline{}{
    \href{https://dl.acm.org/doi/10.1145/3355089.3356537}{A thermomechanical material point method for baking and cooking}. 
    Mengyuan Ding, Xuchen Han, \underline{Stephanie Wang}, Theodore F. Gast, and Joseph M. Teran. SIGGRAPH Asia 2019.
}
\cvline{}{
    \href{https://dl.acm.org/doi/10.1145/3340258}{A hybrid material point method for frictional contact with diverse materials}. 
    Xuchen Han, Theodore F. Gast, Qi Guo, \underline{Stephanie Wang}, Chenfanfu Jiang, and Joseph Teran. SCA 2019.
}
\cvline{}{
    \href{https://dl.acm.org/doi/10.1145/3340259}{Simulation and visualization of ductile fracture with the material point method}. 
    \underline{Stephanie Wang}, Mengyuan Ding, Theodore F. Gast, Leyi Zhu, Steven Gagniere, Chenfanfu Jiang, and Joseph M. Teran. SCA 2019 (\textbf{Best Paper Award}).
}
\begin{center}
    Last updated: \today.
\end{center}
\end{document}
%%%%%%%%%%%%%%%%%%%%%%%%%%%%%%%%%%%%%%%%%%%%%%%%%%%%%%%%%%%%%%%%%%%%%%%%%%%%

\begin{itemize}[leftmargin=1cm,itemsep=0.3ex]
\item % \bibentry{Nabizadeh:2022} 
    \href{https://dl.acm.org/doi/10.1145/3528223.3530120}{Covector fluids}. 
    Mohammad Sina Nabizadeh, \underline{Stephanie Wang}, Ravi Ramamoorthi, and Albert Chern. SIGGRAPH 2022.
\item % \bibentry{Palmer:2021} 
    \href{https://openaccess.thecvf.com/content/CVPR2022/papers/Palmer_DeepCurrents_Learning_Implicit_Representations_of_Shapes_With_Boundaries_CVPR_2022_paper.pdf}{DeepCurrents: Learning implicit representations of shapes with boundaries}. 
    David Palmer, Dmitriy Smirnov, \underline{Stephanie Wang}, Albert Chern, and Justin Solomon. CVPR 2022.
\item % \bibentry{Wang:2021}
    \href{https://dl.acm.org/doi/10.1145/3450626.3459781}{Computing minimal surfaces with differential forms}. 
    \underline{Stephanie Wang} and Albert Chern. SIGGRAPH 2021.
\item % \bibentry{Ding:2019}
    \href{https://dl.acm.org/doi/10.1145/3355089.3356537}{A thermomechanical material point method for baking and cooking}. 
    Mengyuan Ding, Xuchen Han, \underline{Stephanie Wang}, Theodore F. Gast, and Joseph M. Teran. SIGGRAPH Asia 2019.
\item % \bibentry{Han:2019}
    \href{https://dl.acm.org/doi/10.1145/3340258}{A hybrid material point method for frictional contact with diverse materials}. 
    Xuchen Han, Theodore F. Gast, Qi Guo, \underline{Stephanie Wang}, Chenfanfu Jiang, and Joseph Teran. SCA 2019.
\item % \bibentry{Wang:2019}
    \href{https://dl.acm.org/doi/10.1145/3340259}{Simulation and visualization of ductile fracture with the material point method}. 
    \underline{Stephanie Wang}, Mengyuan Ding, Theodore F. Gast, Leyi Zhu, Steven Gagniere, Chenfanfu Jiang, and Joseph M. Teran. SCA 2019 (\textbf{Best Paper Award}).
\end{itemize}
