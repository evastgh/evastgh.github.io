\documentclass[11pt,a4paper,unicode]{moderncv}

\moderncvtheme{classic}
\usepackage[utf8]{inputenc}

\usepackage[top=.55in,bottom=.55in,left=.7in, right=.7in]{geometry}          % smaller number -> text takes more space
\AtBeginDocument{\recomputelengths}

% The following lines include both bibentry and hyperref packages while resolving the clash
% See https://tex.stackexchange.com/questions/65348/clash-between-bibentry-and-hyperref-with-bibstyle-elsart-harv
\usepackage{bibentry} % using BibTeX for publication list
\makeatletter\let\saved@bibitem\@bibitem\makeatother
\usepackage[unicode]{hyperref} % hyperlinks
\makeatletter\let\@bibitem\saved@bibitem\makeatother

\definecolor{linkcolour}{rgb}{0,0.2,0.6}            % hyperlinks setup
\hypersetup{colorlinks,breaklinks,urlcolor=linkcolour, linkcolor=linkcolour}

\usepackage{enumitem} %customization for enumerate

\firstname{Stephanie}
\familyname{Wang}
% \address{University of California, San Diego}{Department of Computer Science and Engineering}
\email{evast@g.ucla.edu}
\homepage{https://evastgh.github.io/}


\nopagenumbers{}                    

\begin{document}
\maketitle\vspace*{-13mm}

\section{Education}
%\cventry{years}{degree/job title}{institution/employer}{localization}{grade}{description}
\cventry{Mar 2020}{Ph.D. in Mathematics}{UCLA}{3.88/4}{Dissertation advisor: Prof. Joseph Teran}
{}
% \cventry{Jun 2016}{M.S. in Mathematics}{UCLA}{}{} {} % {As a part of the PhD program.}
\cventry{Jan 2013}{B.S. in Mathematics}{National Taiwan University}{}{3.64/4 \textit{magna cum laude}}
{} % {Dean's Award of College of Science \textit{(magna cum laude)} \\ 3\textsuperscript{rd} place in Applied and Computational Mathematics in Yau's College Student Mathematics Contest}

\vspace*{-3mm}
\section{Positions}
\vspace*{-2mm}
\subsection{Research experience}
\vspace*{-1mm}
\cventry{2020-present}{Postdoc -- under Prof. \href{https://cseweb.ucsd.edu/~alchern/}{Albert Chern}}{UCSD}{}{San Diego, CA}{Geometry processing, physical simulation, inverse rendering, and geometry learning. Mentored students: 
    \href{https://sinabiz.github.io/}{Mohammad Sina Nabizadeh},
    \href{https://www.linkedin.com/in/shiyang-jia-a4aa6b174}{Shiyang Jia},
    \href{https://chadmckell.com/}{Chad McKell}.
}
\cventry{2019-2020}{Ph.D. Study -- under Prof. \href{about:blank}{Wilfrid Gangbo}}{UCLA}{}{Los Angeles, CA}{Regularity theory for minimizers of polyconvex functionals related to Navier-Stokes equation. }
\cventry{2019 summer}{Summer Exchange -- under Prof. \href{https://people.epfl.ch/johan.gaume}{Johan Gaume}}{EPFL}{}{Lausanne, Switzerland}
{Physics-based simulations, post-processing, and data analysis of snow and tire interaction and consulting at the Snow and Avalanche Simulation Laboratory. }
\cventry{2016-2019}{Ph.D Study -- under Prof. \href{https://www.math.ucla.edu/~jteran/}{Joseph Teran}}{UCLA}{}{Los Angeles, CA}
{Physics-based simulations for animation purposes using C++ programming, convex and nonconvex optimization, numerical PDEs, numerical linear algebra, parallel computing. }
\cventry{2013-2014}{Research Assistant -- under Prof. \href{https://scholar.nycu.edu.tw/en/persons/wen-wei-lin}{Wen-Wei Lin}}{NCTU}{}{Hsinchu, Taiwan}{Generalized eigenvalue problems using MATLAB programming.}

\vspace*{-1mm}
\subsection{Industry Experience}
\vspace*{-1mm}
\cventry{2018 summer}{Technology Intern}{Walt Disney Animation Studio}{}{Burbank, CA}
{R\&D for pioneering simulation technology in animated feature film, teaming with FX artists, numerical analysis, continuum mechanics, C++, HDK.}

\vspace*{-1mm}
\subsection{Teaching Experience}
\vspace*{-1mm}
\cventry{2019-2020}{Assistant Adjunct Professor / Instructor}{UCLA Math Dept}{}{Los Angeles, CA}
{Taught courses: linear algebra, machine learning (remote) and multivariable calculus (remote). }
% \cventry{2019 spring}{Instructor}{UCLA Math Dept}{}{Los Angeles, CA}
% {Taught course: linear algebra and applications. }
\cventry{2015-2020}{Teaching Assistant}{UCLA Math Dept}{}{Los Angeles, CA}
{Taught course: linear algebra and intro to mathematical proofs, undergrad and grad level numerical methods, intro, intermediate, and advanced C++ programming.}
% \cventry{2014 summer}{Course Organizer}{Formosan Summer School on Logic, Language, and Computation}{}{}{} % {Organized 2014 Formosan Summer School on Logic, Language, and Computation.}


\vspace*{-2mm}
\section{Skills}
\cvline{Programming}{C++ (Eigen, tbb, gdb, valgrind), lua, MATLAB (CVX), zsh, \LaTeX, Houdini, Vim, git}
\cvline{Mathematics}{Optimization, differential geometry, numerical and theoretical PDEs, scientific computing.}
\cvline{Languages}{English and Mandarin Chinese - bilingual proficiency.}{}
\cvline{Hobbies}{Rock climbing, hiking, and cooking}


\nocite{*}
\bibliographystyle{plain}
\nobibliography{publications}

\vspace*{-2.5mm}
\section{Selected Publications}
\begin{itemize}[leftmargin=1cm,itemsep=0.3ex]
    \small
    \item \bibentry{Blatny:2021}
    \href{https://www.sciencedirect.com/science/article/pii/S0266352X21002822}{(ScienceDirect)}
    \item \bibentry{Wang:2021}
        \href{https://dl.acm.org/doi/10.1145/3450626.3459781}{(ACM Digital Library)}
    % \item \bibentry{Wang:2020}
    \item \bibentry{Ding:2019}
    \href{https://dl.acm.org/doi/10.1145/3355089.3356537}{(ACM Digital Library)}
    \item \bibentry{Han:2019}
    \href{https://dl.acm.org/doi/10.1145/3340258}{(ACM Digital Library)} 
    \item \bibentry{Wang:2019}
    \href{https://dl.acm.org/doi/10.1145/3340259}{(ACM Digital Library)}
\end{itemize}
% \cvline{Aug 2021}{\textbf{Stephanie Wang} and Albert Chern, Computing Minimal Surfaces with Differential Forms, ACM Transactions on Graphics (SIGGRAPH 2021)}
% \cvline{Mar 2020}{\textbf{Stephanie Wang}, A Material Point Method for Elastoplasticity with Ductile Fracture and Frictional Contact, Doctoral Dissertation, UCLA.}
% \cvline{Nov 2019}{M. Ding, X. Han, \textbf{S. Wang}, T. Gast, J. Teran, A thermomechanical material point method for baking and cooking, ACM Transactions on Graphics (SIGGRAPH Asia 2019)}
% \cvline{Jul 2019}{X. Han, T. Gast, Q. Guo, \textbf{S. Wang}, C. Jiang, J. Teran, A Hybrid Material Point Method for Frictional Contact with Diverse Materials, Proc ACM on Computer Graphics and Interactive Techniques (SCA 2019)}
% \cvline{Jul 2019}{\textbf{S. Wang}, M. Ding, T. Gast, L. Zhu, S. Gagniere, C. Jiang, J. Teran, Simulation and Visualization of Ductile Fracture with the Material Point Method, Proc ACM on Computer Graphics and Interactive Techniques (SCA 2019 Best Paper)}
% \cvline{Aug 2019}{J. Carlen, J. Pont, C. Mentus, S. Chang, \textbf{S. Wang}, M. Porter, Role Detection in Bicycle-Sharing Networks Using Multilayer Stochastic Block Models, arXiv:1908.09440}

\end{document}
