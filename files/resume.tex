\documentclass[11pt,a4paper,unicode]{moderncv}

\moderncvtheme{classic}
\usepackage[utf8]{inputenc}

\usepackage[top=.55in,bottom=.55in,left=.7in, right=.7in]{geometry}          % smaller number -> text takes more space
\AtBeginDocument{\recomputelengths}

% Hyperlinks
\usepackage[unicode]{hyperref}                      % to use hyperlinks
\definecolor{linkcolour}{rgb}{0,0.2,0.6}            % hyperlinks setup
\hypersetup{colorlinks,breaklinks,urlcolor=linkcolour, linkcolor=linkcolour}

\firstname{Stephanie}
\familyname{Wang}
% \address{}{Los Angeles, CA}
\email{evast@g.ucla.edu}


\nopagenumbers{}                    

\begin{document}
\maketitle\vspace*{-8mm}

\section{Education}
%\cventry{years}{degree/job title}{institution/employer}{localization}{grade}{description}
\cventry{Mar 2020}{Ph.D. in Mathematics}{UCLA}{3.88/4}{Dissertation advisor: Prof. Joseph Teran}
{}
% \cventry{Jun 2016}{M.S. in Mathematics}{UCLA}{}{} {} % {As a part of the PhD program.}
\cventry{Jan 2013}{B.S. in Mathematics}{National Taiwan University}{}{3.64/4 \textit{magna cum laude}}
{} % {Dean's Award of College of Science \textit{(magna cum laude)} \\ 3\textsuperscript{rd} place in Applied and Computational Mathematics in Yau's College Student Mathematics Contest}


\section{Research Experience}
\cventry{2020-present}{Postdoc -- under Prof. Albert Chern}{UCSD}{}{San Diego, CA}{Weird maths with applications in graphics using Houdini and python programming.}
\cventry{2019-2020}{Ph.D. Study -- under Prof. Wilfrid Gangbo}{UCLA}{}{Los Angeles, CA}{Regularity theory for minimizers of polyconvex functionals related to Navier-Stokes equation. }
\cventry{2019 summer}{Summer Exchange -- under Prof. Johan Gaume}{EPFL}{}{Lausanne, Switzerland}
{Physics-based simulations, post-processing, and data analysis of snow and tire interaction and consulting at the Snow and Avalanche Simulation Laboratory. }
\cventry{2016-2019}{Ph.D Study -- under Prof. Joseph Teran}{UCLA}{}{Los Angeles, CA}
{Physics-based simulations for animation purposes using C++ programming, convex and nonconvex optimization, numerical PDEs, numerical linear algebra, parallel computing. }
\cventry{2013-2014}{Research Assistant -- under Prof. Wen-Wei Lin}{NCTU}{}{Hsinchu, Taiwan}{Generalized eigenvalue problems using MATLAB programming.}


\section{Employment}
\cventry{2020}{Assistant Adjunct Professor}{UCLA Math Dept}{}{Los Angeles, CA}
{Teaching Machine Learning (Math156) and Calculus of Several Variables (Math32A).}
\cventry{2019 spring}{Instructor}{UCLA Math Dept}{}{Los Angeles, CA}
{Teaching Linear Algebra and Applications (Math33A).}
\cventry{2018 summer}{Technology Intern}{Walt Disney Animation Studio}{}{Burbank, CA}
{R\&D for pioneer simulation technology in animated feature film, teaming with FX artists, numerical analysis, continuum mechanics, C++, HDK.}
\cventry{2015-2020}{Teaching Assistant}{UCLA Math Dept}{}{Los Angeles, CA}
{Linear algebra and intro to mathematical proofs, undergrad and grad level numerical methods, intro, intermediate, and advanced C++ programming.}
% \cventry{2014 summer}{Course Organizer}{Formosan Summer School on Logic, Language, and Computation}{}{}{} % {Organized 2014 Formosan Summer School on Logic, Language, and Computation.}


\section{Skills}
\cvline{Programming}{C++ (Eigen, tbb, gdb, valgrind), lua, MATLAB (CVX), zsh, \LaTeX, Houdini, Vim, git}
\cvline{Mathematics}{Optimization, differential equations, scientific computing, and numerical linear algebra.}
\cvline{Languages}{English and Mandarin Chinese - bilingual proficiency.}{}
\cvline{Hobbies}{Rock climbing, hiking, and cooking}


\section{Selected Publications}
\nocite{*}
\bibliographystyle{plain}
\bibliography{publications}
% \cvline{Mar 2020}{\textbf{Stephanie Wang}, A Material Point Method for Elastoplasticity with Ductile Fracture and Frictional Contact, Doctoral Dissertation, UCLA.}
\cvline{Aug 2021}{\textbf{Stephanie Wang} and Albert Chern, Computing Minimal Surfaces with Differential Forms, ACM Transactions on Graphics (SIGGRAPH 2021)}
\cvline{Nov 2019}{M. Ding, X. Han, \textbf{S. Wang}, T. Gast, J. Teran, A thermomechanical material point method for baking and cooking, ACM Transactions on Graphics (SIGGRAPH Asia 2019)}
\cvline{Jul 2019}{X. Han, T. Gast, Q. Guo, \textbf{S. Wang}, C. Jiang, J. Teran, A Hybrid Material Point Method for Frictional Contact with Diverse Materials, Proc ACM on Computer Graphics and Interactive Techniques (SCA 2019)}
\cvline{Jul 2019}{\textbf{S. Wang}, M. Ding, T. Gast, L. Zhu, S. Gagniere, C. Jiang, J. Teran, Simulation and Visualization of Ductile Fracture with the Material Point Method, Proc ACM on Computer Graphics and Interactive Techniques (SCA 2019 Best Paper)}
% \cvline{Aug 2019}{J. Carlen, J. Pont, C. Mentus, S. Chang, \textbf{S. Wang}, M. Porter, Role Detection in Bicycle-Sharing Networks Using Multilayer Stochastic Block Models, arXiv:1908.09440}

\end{document}
